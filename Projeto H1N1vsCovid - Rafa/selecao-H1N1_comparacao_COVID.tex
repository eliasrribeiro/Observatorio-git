% Options for packages loaded elsewhere
\PassOptionsToPackage{unicode}{hyperref}
\PassOptionsToPackage{hyphens}{url}
%
\documentclass[
]{article}
\usepackage{amsmath,amssymb}
\usepackage{lmodern}
\usepackage{ifxetex,ifluatex}
\ifnum 0\ifxetex 1\fi\ifluatex 1\fi=0 % if pdftex
  \usepackage[T1]{fontenc}
  \usepackage[utf8]{inputenc}
  \usepackage{textcomp} % provide euro and other symbols
\else % if luatex or xetex
  \usepackage{unicode-math}
  \defaultfontfeatures{Scale=MatchLowercase}
  \defaultfontfeatures[\rmfamily]{Ligatures=TeX,Scale=1}
\fi
% Use upquote if available, for straight quotes in verbatim environments
\IfFileExists{upquote.sty}{\usepackage{upquote}}{}
\IfFileExists{microtype.sty}{% use microtype if available
  \usepackage[]{microtype}
  \UseMicrotypeSet[protrusion]{basicmath} % disable protrusion for tt fonts
}{}
\makeatletter
\@ifundefined{KOMAClassName}{% if non-KOMA class
  \IfFileExists{parskip.sty}{%
    \usepackage{parskip}
  }{% else
    \setlength{\parindent}{0pt}
    \setlength{\parskip}{6pt plus 2pt minus 1pt}}
}{% if KOMA class
  \KOMAoptions{parskip=half}}
\makeatother
\usepackage{xcolor}
\IfFileExists{xurl.sty}{\usepackage{xurl}}{} % add URL line breaks if available
\IfFileExists{bookmark.sty}{\usepackage{bookmark}}{\usepackage{hyperref}}
\hypersetup{
  pdftitle={Seleção dos casos de H1N1 para paper de comparação com COVID-19},
  pdfauthor={Gestantes},
  hidelinks,
  pdfcreator={LaTeX via pandoc}}
\urlstyle{same} % disable monospaced font for URLs
\usepackage[margin=1in]{geometry}
\usepackage{color}
\usepackage{fancyvrb}
\newcommand{\VerbBar}{|}
\newcommand{\VERB}{\Verb[commandchars=\\\{\}]}
\DefineVerbatimEnvironment{Highlighting}{Verbatim}{commandchars=\\\{\}}
% Add ',fontsize=\small' for more characters per line
\usepackage{framed}
\definecolor{shadecolor}{RGB}{248,248,248}
\newenvironment{Shaded}{\begin{snugshade}}{\end{snugshade}}
\newcommand{\AlertTok}[1]{\textcolor[rgb]{0.94,0.16,0.16}{#1}}
\newcommand{\AnnotationTok}[1]{\textcolor[rgb]{0.56,0.35,0.01}{\textbf{\textit{#1}}}}
\newcommand{\AttributeTok}[1]{\textcolor[rgb]{0.77,0.63,0.00}{#1}}
\newcommand{\BaseNTok}[1]{\textcolor[rgb]{0.00,0.00,0.81}{#1}}
\newcommand{\BuiltInTok}[1]{#1}
\newcommand{\CharTok}[1]{\textcolor[rgb]{0.31,0.60,0.02}{#1}}
\newcommand{\CommentTok}[1]{\textcolor[rgb]{0.56,0.35,0.01}{\textit{#1}}}
\newcommand{\CommentVarTok}[1]{\textcolor[rgb]{0.56,0.35,0.01}{\textbf{\textit{#1}}}}
\newcommand{\ConstantTok}[1]{\textcolor[rgb]{0.00,0.00,0.00}{#1}}
\newcommand{\ControlFlowTok}[1]{\textcolor[rgb]{0.13,0.29,0.53}{\textbf{#1}}}
\newcommand{\DataTypeTok}[1]{\textcolor[rgb]{0.13,0.29,0.53}{#1}}
\newcommand{\DecValTok}[1]{\textcolor[rgb]{0.00,0.00,0.81}{#1}}
\newcommand{\DocumentationTok}[1]{\textcolor[rgb]{0.56,0.35,0.01}{\textbf{\textit{#1}}}}
\newcommand{\ErrorTok}[1]{\textcolor[rgb]{0.64,0.00,0.00}{\textbf{#1}}}
\newcommand{\ExtensionTok}[1]{#1}
\newcommand{\FloatTok}[1]{\textcolor[rgb]{0.00,0.00,0.81}{#1}}
\newcommand{\FunctionTok}[1]{\textcolor[rgb]{0.00,0.00,0.00}{#1}}
\newcommand{\ImportTok}[1]{#1}
\newcommand{\InformationTok}[1]{\textcolor[rgb]{0.56,0.35,0.01}{\textbf{\textit{#1}}}}
\newcommand{\KeywordTok}[1]{\textcolor[rgb]{0.13,0.29,0.53}{\textbf{#1}}}
\newcommand{\NormalTok}[1]{#1}
\newcommand{\OperatorTok}[1]{\textcolor[rgb]{0.81,0.36,0.00}{\textbf{#1}}}
\newcommand{\OtherTok}[1]{\textcolor[rgb]{0.56,0.35,0.01}{#1}}
\newcommand{\PreprocessorTok}[1]{\textcolor[rgb]{0.56,0.35,0.01}{\textit{#1}}}
\newcommand{\RegionMarkerTok}[1]{#1}
\newcommand{\SpecialCharTok}[1]{\textcolor[rgb]{0.00,0.00,0.00}{#1}}
\newcommand{\SpecialStringTok}[1]{\textcolor[rgb]{0.31,0.60,0.02}{#1}}
\newcommand{\StringTok}[1]{\textcolor[rgb]{0.31,0.60,0.02}{#1}}
\newcommand{\VariableTok}[1]{\textcolor[rgb]{0.00,0.00,0.00}{#1}}
\newcommand{\VerbatimStringTok}[1]{\textcolor[rgb]{0.31,0.60,0.02}{#1}}
\newcommand{\WarningTok}[1]{\textcolor[rgb]{0.56,0.35,0.01}{\textbf{\textit{#1}}}}
\usepackage{graphicx}
\makeatletter
\def\maxwidth{\ifdim\Gin@nat@width>\linewidth\linewidth\else\Gin@nat@width\fi}
\def\maxheight{\ifdim\Gin@nat@height>\textheight\textheight\else\Gin@nat@height\fi}
\makeatother
% Scale images if necessary, so that they will not overflow the page
% margins by default, and it is still possible to overwrite the defaults
% using explicit options in \includegraphics[width, height, ...]{}
\setkeys{Gin}{width=\maxwidth,height=\maxheight,keepaspectratio}
% Set default figure placement to htbp
\makeatletter
\def\fps@figure{htbp}
\makeatother
\setlength{\emergencystretch}{3em} % prevent overfull lines
\providecommand{\tightlist}{%
  \setlength{\itemsep}{0pt}\setlength{\parskip}{0pt}}
\setcounter{secnumdepth}{-\maxdimen} % remove section numbering
\usepackage{booktabs}
\usepackage{longtable}
\usepackage{array}
\usepackage{multirow}
\usepackage{wrapfig}
\usepackage{float}
\usepackage{colortbl}
\usepackage{pdflscape}
\usepackage{tabu}
\usepackage{threeparttable}
\usepackage{threeparttablex}
\usepackage[normalem]{ulem}
\usepackage{makecell}
\usepackage{xcolor}
\ifluatex
  \usepackage{selnolig}  % disable illegal ligatures
\fi

\title{Seleção dos casos de H1N1 para paper de comparação com COVID-19}
\author{Gestantes}
\date{18/06/2021}

\begin{document}
\maketitle

\hypertarget{sobre-a-base-de-dados-e-pacotes-do-r-utilizados}{%
\section{Sobre a base de dados e pacotes do R
utilizados}\label{sobre-a-base-de-dados-e-pacotes-do-r-utilizados}}

A seguir são carregados os pacotes do R
(\url{https://www.r-project.org}) utilizados para filtragem e tratamento
dos dados.

\begin{Shaded}
\begin{Highlighting}[]
\CommentTok{\#carregar pacotes}
\NormalTok{loadlibrary }\OtherTok{\textless{}{-}} \ControlFlowTok{function}\NormalTok{(x) \{}
  \ControlFlowTok{if}\NormalTok{ (}\SpecialCharTok{!}\FunctionTok{require}\NormalTok{(x, }\AttributeTok{character.only =} \ConstantTok{TRUE}\NormalTok{)) \{}
    \FunctionTok{install.packages}\NormalTok{(x, }\AttributeTok{dependencies =}\NormalTok{ T)}
    \ControlFlowTok{if}\NormalTok{ (}\SpecialCharTok{!}\FunctionTok{require}\NormalTok{(x, }\AttributeTok{character.only =} \ConstantTok{TRUE}\NormalTok{))}
      \FunctionTok{stop}\NormalTok{(}\StringTok{"Package not found"}\NormalTok{)}
\NormalTok{  \}}
\NormalTok{\}}

\NormalTok{packages }\OtherTok{\textless{}{-}}
  \FunctionTok{c}\NormalTok{(}
    \StringTok{"dplyr"}\NormalTok{,}
    \StringTok{"lubridate"}\NormalTok{,}
    \StringTok{"readr"}\NormalTok{,}
    \StringTok{"ggplot2"}\NormalTok{,}
    \StringTok{"kableExtra"}\NormalTok{,}
    \StringTok{"tables"}\NormalTok{,}
    \StringTok{"questionr"}\NormalTok{,}
    \StringTok{"car"}\NormalTok{,}
    \StringTok{"data.table"}\NormalTok{,}
    \StringTok{"magrittr"}\NormalTok{,}
    \StringTok{"tidyverse"}\NormalTok{,}
    \StringTok{"readxl"}\NormalTok{,}
    \StringTok{"summarytools"}\NormalTok{,}
    \StringTok{"zoo"}\NormalTok{,}
    \StringTok{"grid"}\NormalTok{,}
    \StringTok{"gridExtra"}\NormalTok{,}
    \StringTok{"cowplot"}
\NormalTok{  )}
\FunctionTok{lapply}\NormalTok{(packages, loadlibrary)}
\end{Highlighting}
\end{Shaded}

A base de dados SIVEP-Gripe (Sistema de Informação da Vigilância
Epidemiológica da Gripe) tem os registros dos casos e óbitos de SRAG
(Síndrome Respiratória Aguda Grave). A notificação é compulsória para
síndrome gripal (caracterizado por pelo menos dois dos seguintes sinais
e sintomas: febre, mesmo que referida, calafrios, dor de garganta, dor
de cabeça, tosse, coriza, distúrbios olfatórios ou de paladar) e que tem
dispneia / desconforto respiratório ou pressão persistente no peito ou
Saturação de O2 menor que 95\% no ar ambiente ou cor azulada dos lábios
ou rosto. Indivíduos assintomáticos com confirmação laboratorial por
biologia molecular ou exame imunológico para infecção por COVID-19
também são relatados.

Para notificações no Sivep-Gripe, os casos hospitalizados em hospitais
públicos e privados e todas as mortes devido a infecções respiratórias
agudas graves, independentemente da hospitalização, devem ser
considerados.

A vigilância da SRAG no Brasil é desenvolvida pelo Ministério da Saúde
(MS), por meio da Secretaria de Vigilância em Saúde (SVS), desde a
pandemia de Influenza A (H1N1) em 2009. Mais informações em
\url{https://coronavirus.saude.gov.br/definicao-de-caso-e-notificacao}.

O período analisado compreende de dados epidemiológicos de 2009 e 2010,
com banco de dados obtido em 05/05/2021 no site
\url{https://opendatasus.saude.gov.br/dataset/bd-srag-2009-a-2012}. Os
dados de 2009 e de 2010 são carregados e combinados abaixo:

\begin{Shaded}
\begin{Highlighting}[]
\DocumentationTok{\#\#\#\#\#\#\#\#\# carregando as bases de dados \#\#\#\#\#\#\#\#\#\#\#}
\CommentTok{\#2009 e 2010}
\NormalTok{dados2009 }\OtherTok{\textless{}{-}} \FunctionTok{read\_delim}\NormalTok{(}
  \StringTok{"influd09\_limpo{-}final.csv"}\NormalTok{,}
  \StringTok{";"}\NormalTok{,}
  \AttributeTok{escape\_double =} \ConstantTok{FALSE}\NormalTok{,}
  \AttributeTok{locale =} \FunctionTok{locale}\NormalTok{(}\AttributeTok{encoding =} \StringTok{"ISO{-}8859{-}2"}\NormalTok{),}
  \AttributeTok{trim\_ws =} \ConstantTok{TRUE}
\NormalTok{)}

\NormalTok{dados2010 }\OtherTok{\textless{}{-}} \FunctionTok{read\_delim}\NormalTok{(}
  \StringTok{"influd10\_limpo{-}final.csv"}\NormalTok{,}
  \StringTok{";"}\NormalTok{,}
  \AttributeTok{escape\_double =} \ConstantTok{FALSE}\NormalTok{,}
  \AttributeTok{locale =} \FunctionTok{locale}\NormalTok{(}\AttributeTok{encoding =} \StringTok{"ISO{-}8859{-}2"}\NormalTok{),}
  \AttributeTok{trim\_ws =} \ConstantTok{TRUE}
\NormalTok{)}

\NormalTok{dados\_2009 }\OtherTok{\textless{}{-}}\NormalTok{ dados2009 }\SpecialCharTok{\%\textgreater{}\%} 
  \FunctionTok{rename}\NormalTok{(}\AttributeTok{SRAGFINAL =}\NormalTok{ SRAG2009FINAL)}

\NormalTok{dados\_2010 }\OtherTok{\textless{}{-}}\NormalTok{ dados2010 }\SpecialCharTok{\%\textgreater{}\%} 
  \FunctionTok{rename}\NormalTok{(}\AttributeTok{SRAGFINAL =}\NormalTok{ SRAG2010FINAL)}

\NormalTok{dados }\OtherTok{\textless{}{-}} \FunctionTok{full\_join}\NormalTok{(dados\_2009, dados\_2010)}

\NormalTok{sem }\OtherTok{\textless{}{-}} \DecValTok{19}

\FunctionTok{memory.limit}\NormalTok{(}\DecValTok{999999}\NormalTok{)}

\CommentTok{\#Criar variavel de ano do caso}
\NormalTok{dados }\OtherTok{\textless{}{-}}\NormalTok{  dados }\SpecialCharTok{\%\textgreater{}\%}
\NormalTok{  dplyr}\SpecialCharTok{::}\FunctionTok{mutate}\NormalTok{(}
    \AttributeTok{dt\_sint =} \FunctionTok{as.Date}\NormalTok{(DT\_SIN\_PRI, }\AttributeTok{format =} \StringTok{"\%d/\%m/\%Y"}\NormalTok{),}
    \AttributeTok{ano =}\NormalTok{ lubridate}\SpecialCharTok{::}\FunctionTok{year}\NormalTok{(dt\_sint),}
    \AttributeTok{mes =}\NormalTok{ lubridate}\SpecialCharTok{::}\FunctionTok{month}\NormalTok{(dt\_sint)}
\NormalTok{  )}
\end{Highlighting}
\end{Shaded}

Há atualmente 217167 observações na base de dados e são as variáveis:

\begin{Shaded}
\begin{Highlighting}[]
\FunctionTok{names}\NormalTok{(dados)}
\end{Highlighting}
\end{Shaded}

\begin{verbatim}
##   [1] "DT_NOTIFIC" "ID_MUNICIP" "SEM_NOT"    "NU_ANO"     "SG_UF_NOT" 
##   [6] "DT_SIN_PRI" "DT_NASC"    "NU_IDADE_N" "CS_SEXO"    "CS_GESTANT"
##  [11] "CS_RACA"    "CS_ESCOL_N" "SG_UF"      "ID_MN_RESI" "ID_OCUPA_N"
##  [16] "VACINA"     "FEBRE"      "TOSSE"      "CALAFRIO"   "DISPNEIA"  
##  [21] "GARGANTA"   "ARTRALGIA"  "MIALGIA"    "CONJUNTIV"  "CORIZA"    
##  [26] "DIARREIA"   "OUTRO_SIN"  "OUTRO_DES"  "CARDIOPATI" "PNEUMOPATI"
##  [31] "RENAL"      "HEMOGLOBI"  "IMUNODEPRE" "TABAGISMO"  "METABOLICA"
##  [36] "OUT_MORBI"  "MORB_DESC"  "HOSPITAL"   "DT_INTERNA" "CO_UF_INTE"
##  [41] "CO_MU_INTE" "DT_PCR"     "PCR_AMOSTR" "PCR_OUT"    "PCR_RES"   
##  [46] "PCR_ETIOL"  "PCR_TIPO_H" "PCR_TIPO_N" "DT_CULTURA" "CULT_AMOST"
##  [51] "CULT_OUT"   "CULT_RES"   "DT_HEMAGLU" "HEMA_RES"   "HEMA_ETIOL"
##  [56] "HEM_TIPO_H" "HEM_TIPO_N" "DT_RAIOX"   "RAIOX_RES"  "RAIOX_OUT" 
##  [61] "CLASSI_FIN" "CLASSI_OUT" "CRITERIO"   "TPAUTOCTO"  "DOENCA_TRA"
##  [66] "EVOLUCAO"   "DT_OBITO"   "DT_ENCERRA" "DT_DIGITA"  "MONITORA"  
##  [71] "SRAGFINAL"  "OBES_IMC"   "OUT_AMOST"  "DS_OAGEETI" "DS_OUTMET" 
##  [76] "DS_OUTSUB"  "OUT_ANTIV"  "DT_COLETA"  "DT_ENTUTI"  "DT_ANTIVIR"
##  [81] "DT_IFI"     "DT_OUTMET"  "DT_SAIDUTI" "RES_ADNO"   "AMOSTRA"   
##  [86] "HEPATICA"   "NEUROLOGIC" "OBESIDADE"  "PUERPERA"   "SIND_DOWN" 
##  [91] "RES_FLUA"   "RES_FLUB"   "UTI"        "IFI"        "PCR"       
##  [96] "RES_OUTRO"  "OUT_METODO" "RES_PARA1"  "RES_PARA2"  "RES_PARA3" 
## [101] "DESC_RESP"  "SATURACAO"  "ST_TIPOFI"  "TIPO_PCR"   "ANTIVIRAL" 
## [106] "SUPORT_VEN" "RES_VSR"    "RES_FLUASU" "AVE_10_DIA" "dt_sint"   
## [111] "ano"        "mes"
\end{verbatim}

Para ver o dicionário das variáveis, acesse:
\url{https://opendatasus.saude.gov.br/dataset/c9a8f286-44bc-444e-94b4-f4ceded3af2c/resource/8e4ee33a-a7bd-42d1-9505-bbf0eb7e6141/download/dic_dados_influenza-pandemica_antigo.pdf-2009-a-set-2012.pdf}

\hypertarget{filtragem-e-tratamento-dos-dados-para-projeto}{%
\section{Filtragem e tratamento dos dados para
projeto}\label{filtragem-e-tratamento-dos-dados-para-projeto}}

A primeira filtragem consiste em selecionar os casos de maio de 2009 até
maio de 2010.

\begin{Shaded}
\begin{Highlighting}[]
\CommentTok{\#filtrando só os casos de maio de 2020 até maio de 2010}
\NormalTok{dados1 }\OtherTok{\textless{}{-}} \FunctionTok{filter}\NormalTok{(dados, }
\NormalTok{                 (mes }\SpecialCharTok{\textgreater{}=}\DecValTok{5} \SpecialCharTok{\&}\NormalTok{ ano }\SpecialCharTok{==} \DecValTok{2009}\NormalTok{) }\SpecialCharTok{|}\NormalTok{ (mes }\SpecialCharTok{\textless{}=}\DecValTok{5} \SpecialCharTok{\&}\NormalTok{ ano }\SpecialCharTok{==} \DecValTok{2010}\NormalTok{))}
\end{Highlighting}
\end{Shaded}

Há 212584 observações na base de dados.

A próxima seleção será de pessoas do sexo feminino:

\begin{Shaded}
\begin{Highlighting}[]
\CommentTok{\#filtrando F}
\NormalTok{dados2 }\OtherTok{\textless{}{-}} \FunctionTok{filter}\NormalTok{(dados1, CS\_SEXO }\SpecialCharTok{==} \StringTok{"F"}\NormalTok{)}
\end{Highlighting}
\end{Shaded}

Há 115357 observações na base de dados.

O próximo passo é filtrar só as mulheres entre 10 e 49 anos.

\begin{Shaded}
\begin{Highlighting}[]
\CommentTok{\# Filtrando pela faixa de idade de interesse}
\NormalTok{dados3 }\OtherTok{\textless{}{-}}\NormalTok{ dados2 }\SpecialCharTok{\%\textgreater{}\%} 
  \FunctionTok{filter}\NormalTok{(NU\_IDADE\_N }\SpecialCharTok{\textgreater{}} \DecValTok{4009} \SpecialCharTok{\&}\NormalTok{ NU\_IDADE\_N }\SpecialCharTok{\textless{}} \DecValTok{4050}\NormalTok{)}
\end{Highlighting}
\end{Shaded}

Há 75717 observações na base de dados.

A próxima seleção são os casos de covid indicado pela variável
\texttt{CLASSI\_FIN}.

\begin{Shaded}
\begin{Highlighting}[]
\FunctionTok{with}\NormalTok{(dados3, }\FunctionTok{freq}\NormalTok{(CLASSI\_FIN))}
\end{Highlighting}
\end{Shaded}

\begin{verbatim}
## Frequencies  
## dados3$CLASSI_FIN  
## Type: Numeric  
## 
##                Freq   % Valid   % Valid Cum.   % Total   % Total Cum.
## ----------- ------- --------- -------------- --------- --------------
##           1   42997    59.723         59.723    56.786         56.786
##           2    1596     2.217         61.940     2.108         58.894
##           3   27346    37.984         99.924    36.116         95.010
##           4      25     0.035         99.958     0.033         95.043
##           9      30     0.042        100.000     0.040         95.083
##        <NA>    3723                              4.917        100.000
##       Total   75717   100.000        100.000   100.000        100.000
\end{verbatim}

\begin{Shaded}
\begin{Highlighting}[]
\NormalTok{dados4 }\OtherTok{\textless{}{-}}\NormalTok{ dados3 }\SpecialCharTok{\%\textgreater{}\%} 
  \FunctionTok{filter}\NormalTok{(CLASSI\_FIN }\SpecialCharTok{==} \DecValTok{1}\NormalTok{)}
\end{Highlighting}
\end{Shaded}

Há 42997 observações na base de dados.

Agora vamos criar a variável se \texttt{CLASSI\_FIN==1} por PCR ou outro
tipo de diagnóstico.

Essa variável é \texttt{pcr\_test}, com as categorias: \texttt{pcr\_pos}
se PCR positivo (CRITERIO == 1 e PCR\_RES == 1) e \texttt{não} caso
contrário.

\begin{Shaded}
\begin{Highlighting}[]
\CommentTok{\#criar a variavel de pcr\_test pelas variaveis de pcr}
\NormalTok{dados4 }\OtherTok{\textless{}{-}}\NormalTok{ dados4 }\SpecialCharTok{\%\textgreater{}\%}
  \FunctionTok{mutate}\NormalTok{(}\AttributeTok{pcr\_test =} \FunctionTok{case\_when}\NormalTok{(PCR\_RES }\SpecialCharTok{==} \DecValTok{1} \SpecialCharTok{\&} \CommentTok{\#pcr positivo }
\NormalTok{                                PCR\_ETIOL}\SpecialCharTok{==} \DecValTok{1}  \SpecialCharTok{\textasciitilde{}} \StringTok{"pcr\_pos"}\NormalTok{,}
                              \ConstantTok{TRUE} \SpecialCharTok{\textasciitilde{}} \StringTok{"não"}
\NormalTok{                              ))}
\end{Highlighting}
\end{Shaded}

Agora iremos filtrar os casos com pcr\_test!=não que são os casos onde
podem ser covid-19 apenas por PCR :

\begin{Shaded}
\begin{Highlighting}[]
\NormalTok{dados5 }\OtherTok{\textless{}{-}}\NormalTok{ dados4 }\SpecialCharTok{\%\textgreater{}\%} 
  \FunctionTok{filter}\NormalTok{(pcr\_test }\SpecialCharTok{==} \StringTok{"pcr\_pos"}\NormalTok{)}
\end{Highlighting}
\end{Shaded}

Há 10659 observações na base de dados.

Agora vamos selecionar só as pessoas gestantes ou não gestante.

\begin{Shaded}
\begin{Highlighting}[]
\FunctionTok{with}\NormalTok{(dados5, }\FunctionTok{freq}\NormalTok{(CS\_GESTANT))}
\end{Highlighting}
\end{Shaded}

\begin{verbatim}
## Frequencies  
## dados5$CS_GESTANT  
## Type: Numeric  
## 
##                Freq   % Valid   % Valid Cum.   % Total   % Total Cum.
## ----------- ------- --------- -------------- --------- --------------
##           1     537      5.04           5.04      5.04           5.04
##           2    1045      9.80          14.84      9.80          14.84
##           3    1126     10.56          25.41     10.56          25.41
##           4      76      0.71          26.12      0.71          26.12
##           5    6070     56.95          83.07     56.95          83.07
##           6     666      6.25          89.31      6.25          89.31
##           9    1139     10.69         100.00     10.69         100.00
##        <NA>       0                               0.00         100.00
##       Total   10659    100.00         100.00    100.00         100.00
\end{verbatim}

\begin{Shaded}
\begin{Highlighting}[]
\NormalTok{dados5 }\OtherTok{\textless{}{-}}\NormalTok{ dados5 }\SpecialCharTok{\%\textgreater{}\%}
  \FunctionTok{mutate}\NormalTok{(}
    \AttributeTok{classi\_gesta =} \FunctionTok{case\_when}\NormalTok{(}
\NormalTok{      CS\_GESTANT }\SpecialCharTok{==} \DecValTok{1}  \SpecialCharTok{\textasciitilde{}} \StringTok{"1tri"}\NormalTok{,}
\NormalTok{      CS\_GESTANT }\SpecialCharTok{==} \DecValTok{2}  \SpecialCharTok{\textasciitilde{}} \StringTok{"2tri"}\NormalTok{,}
\NormalTok{      CS\_GESTANT }\SpecialCharTok{==} \DecValTok{3}  \SpecialCharTok{\textasciitilde{}} \StringTok{"3tri"}\NormalTok{,}
\NormalTok{      CS\_GESTANT }\SpecialCharTok{==} \DecValTok{4}  \SpecialCharTok{\textasciitilde{}} \StringTok{"IG\_ig"}\NormalTok{,}
\NormalTok{      CS\_GESTANT }\SpecialCharTok{==} \DecValTok{5}  \SpecialCharTok{\textasciitilde{}} \StringTok{"não"}\NormalTok{,}
      \ConstantTok{TRUE} \SpecialCharTok{\textasciitilde{}} \ConstantTok{NA\_character\_}
\NormalTok{    )}
\NormalTok{  )}

\CommentTok{\#filtrando só gestante ou não gestante}
\NormalTok{dados6 }\OtherTok{\textless{}{-}}\NormalTok{ dados5 }\SpecialCharTok{\%\textgreater{}\%} 
  \FunctionTok{filter}\NormalTok{(}\SpecialCharTok{!}\FunctionTok{is.na}\NormalTok{(classi\_gesta))}


\CommentTok{\# Criando a variável gestante\_SN}
\NormalTok{dados6 }\OtherTok{\textless{}{-}}\NormalTok{ dados6 }\SpecialCharTok{\%\textgreater{}\%}
  \FunctionTok{mutate}\NormalTok{(}\AttributeTok{gestante\_SN =} \FunctionTok{ifelse}\NormalTok{(CS\_GESTANT }\SpecialCharTok{==} \DecValTok{5}\NormalTok{, }\StringTok{"não"}\NormalTok{, }\StringTok{"sim"}\NormalTok{))}
\end{Highlighting}
\end{Shaded}

Há 8854 observações na base de dados.

\end{document}
